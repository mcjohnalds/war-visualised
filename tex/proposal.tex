\documentclass[12pt]{article}
\usepackage[margin=1in]{geometry}
\usepackage{sectsty}
\usepackage{parskip}

\sectionfont{\fontsize{12}{0}\selectfont}
\pagenumbering{gobble}

\title{COSC3000 Assignment 1 Proposal \vspace{-0.5em}}
\author{John Owen (43591617) \vspace{-0.5em}}
\date{}

\begin{document}

\maketitle

The project aims to visualise how war changes over time, along with the
impact of nations' behaviour and power on military crises. The project is
interesting becuase it uses analytical methods to describe war, whereas
articles and documentaries usually stick to high-level descriptions of war,
devoid of graphs and statistics.

There is a large amount of data available on armed conflicts around the world. I
will attempt to find patterns and trends in the data by analysing and displaying
it through different types of graphs. I might use analysis techniques like
fourier analysis, PCA, and MCA if time permits. The datasets I will use are
listed below.

For this project I will be using python with libraries such as matplotlib
(matlab clone) and numpy (scientific computing).

\section*{UCDP/PRIO Armed Conflict Dataset}

Lists the opposing sides, location, reason for war (government, territory, or
both), type of war (state vs state, state vs non-state group, both) intensity,
date, and duration of 259 conflicts ranging from 1946 to 2014.

\section*{International Crisis Behavior Project}

Describes the involvment of nations, effectiveness of nations' actions,
and outcomes of 470 international crises from 1918 to 2014 where conflict
occured.

\section*{Correlates of War Project: National Material Capabilities}

Details the production cabability, military power, and population of 160
countries over time.

\end{document}
